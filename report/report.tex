\documentclass[12pt]{article}

\usepackage[T2A]{fontenc}
\usepackage[utf8x]{inputenc}
\usepackage[russian]{babel}

\renewcommand{\thepage}{\large\arabic{page}}
\usepackage{amssymb,amsmath,braket}
\usepackage{fancyhdr}
\usepackage[left=3cm,right=1.5cm,top=2cm,bottom=2.5cm]{geometry}
\usepackage{graphicx}
\usepackage{float}
\tolerance=600
\emergencystretch=15pt
\unitlength=0.5mm
\usepackage{latexsym}
\usepackage{ mathrsfs }
\abovecaptionskip=0pt
\belowcaptionskip=0pt
\usepackage{indentfirst}
\parindent=1.27cm
\abovecaptionskip=15mm
\belowcaptionskip=8mm
\addto\captionsrussian{\def\refname{Список  использованных источников}}
\addto\captionrussian{\def\tablename{ }}
\usepackage{titlesec}

\titleformat{\section}[block]{\bfseries\centering}{\thesection}{1ex}{}
\titleformat{\subsection}[block]{\bfseries\large\centering}{\thesubsection}{1ex}{}
\titleformat{\subsubsection}[block]{\bfseries\large\centering}{\thesubsubsection}{1ex}{}
\numberwithin{equation}{section}

\usepackage[usenames]{color}
\usepackage{xcolor}
\usepackage{hyperref}
\definecolor{linkcolor}{HTML}{0F0F0F} % цвет ссылок
\definecolor{urlcolor}{HTML}{008B8B} % цвет гиперссылок

\hypersetup{pdfstartview=FitH,  linkcolor=linkcolor,urlcolor=urlcolor, colorlinks=true}

\usepackage{listings}
\usepackage{color}

%%%%%%%%%%%%%%%%%%%%%%%%%%%%%%%%%%%%%%%%%%%%%%%

\begin{document}

\pagestyle{empty}
 %
\fontsize{14}{17pt} \selectfont
\begin{titlepage}
\begin{center}
{Министерство образования и науки Российской Федерации\\[5pt]}
{федеральное государственное бюджетное образовательное учреждение высшего образования \\[5pt]}
{"<Иркутский государственный университет">\\[5pt]}
{(ФГБОУ ВО "<ИГУ">)\\[5pt]}
Физический факультет \\[35pt]
\begin{flushright}
Кафедра теоретической физики\\
Заведующий кафедрой\\
доцент Ловцов С.В. \underline{\qquad\qquad\qquad}\\[35pt]
\end{flushright}
\Large {Отчет по научно-исследовательской работе\\[20pt]}

{\Large Использование разложения Магнуса для решения уравнения осцилляций нейтрино в среде}\\[6pt]
\bigskip
\end{center}
\begin{flushright}
\begin{minipage}[bf]{8cm}
Руководитель практики:\\[5pt]
\underline{\qquad\qquad\quad\quad} \mbox{\text{к.ф.-м.н. Ломов В. П.}} \\
Студент гр. 01411-ДБ\\[5pt]
\underline{\qquad\qquad\qquad\qquad}  Шайдурова А. В. \\
\mbox{\text{Работа защищена с оценкой\underline{\qquad\qquad}}}\\
Нормоконтролер:\\
\underline{\qquad\qquad\qquad\qquad} Фролова А.С.\\
\end{minipage}
\bigskip
\end{flushright}

\begin{center}
	Иркутск 2018
\end{center}
\end{titlepage}
%%%%%%%%%%%%%%%%%%%%%%%%%%%%%%%%%%%%%%%%%%%%%%%%%%
\newpage
\section*{\large Реферат}
\newpage
\begin{center}
	{\large \mdseries \tableofcontents}
\end{center}
\newpage

\pagestyle{fancy} % смена стиля оформления страниц
\fancyhf{} % очистка текущих значений

\renewcommand{\headrulewidth}{0pt}
\fancyfoot[FR]{\thepage} % установка нижнего колонтитула
\fancypagestyle{plain}{
	\fancyhf{}}
%\rhead{\thepage}}
\setcounter{page}{4} % начать нумерацию страниц с №4

\section*{\large Введение}
\addcontentsline{toc}{section}{\large Введение}\indent
Краткий экскурс в основные понятия осцилляций нейтрино.
\paragraph{\large Нейтринные осцилляции}

В Стандартной Модели физики частиц нейтрино безмассовые. Однако эксперименты показывают, что нейтрино не только имеют массу, но и находятся в состоянии суперпозиции из собственных векторов, отвечающих за аромат (флэйвор) $\nu_{\alpha}, \; \alpha \!\in \!\{e,\,\mu,\, \tau\}$. Флэйворные состояния не идентичны массовым состояниям $\nu_{k}, \; k\!\in\! \{1,2,3\}$.
	
Эта суперпозиция осуществляется с помощью унитарной матрицы смешивания $U$ \footnote {Матрица смешивания $U$ в случае с 3 нейтрино является PMNS-матрицей (Понтекорво-Маки-Накагава-Саката)}
\begin{equation}
\label{mixing}
\ket{\nu_{\alpha}} = \sum_{k} U^*_{\alpha k} \ket{\nu_{k}}, 
\end{equation}
соответственно, массовые состояния могут быть выражены через суперпозицию состояний аромата
\begin{equation}
\ket{\nu_{k}} = \sum_{\alpha} U_{\alpha k} \ket{\nu_{\alpha}},
\end{equation}
и базис флэйворных состояний является ортонормированным, так же, как и базис массовых состояний:
\begin{equation}
\braket{\nu_{\alpha}|\nu_{\beta}} = \delta_{\alpha\beta},
\hspace{2cm} 
\braket{\nu_{k}|\nu_{j}}=\delta_{kj}
\end{equation}
	
В качестве доказательства смешивания было обнаружено, что нейтрино изменяются от одного аромата к другому во время их распространения - явление, называемое нейтринными осцилляциями, которое было предложено Понтекорво в 1957 году по аналогии с осцилляциями K-мезонов.

Матричное уравнение (\ref{mixing}) также записывают в следующем виде
\begin{equation}
\begin{pmatrix}
	\nu_{e}\\
	\nu_{\mu}\\
	\nu_{\tau}\\
\end{pmatrix}
\!\!=\!\!
\begin{pmatrix}
	U_{e1}& U_{e2}& U_{e3}\\
	U_{\mu1}& U_{\mu2}& U_{\mu3}\\
	U_{\tau1}& U_{\tau3}& U_{\tau3}\\
\end{pmatrix}
\!\!\cdot\!\!
\begin{pmatrix}
	\nu_{1}\\
	\nu_{2}\\
	\nu_{3}\\
\end{pmatrix},
\end{equation}
а явный вид матрицы смешивания зависит от угловых параметров
\begin{equation}
U_{P\!M\!N\!S}\! =\! 
\begin{pmatrix}
c_{12}c_{13}& s_{12}s_{13}& s_{13}e^{-i\delta_{CP}}\\
-s_{12}c_{23} - c_{12}s_{23}s_{13}e^{i\delta_{CP}}& c_{12}c_{23} - s_{12}s_{23}s_{13}e^{i\delta_{CP}}& c_{13}s_{23}\\
s_{12}s_{23}-c_{12}c_{23}s_{13}e^{i\delta_{CP}}& -c_{12}s_{23}-s_{12}c_{23}s_{13}e^{i\delta_{CP}}& c_{13}c_{23}\\
\end{pmatrix},
\end{equation}
где $s_{ij}=sin\theta_{ij}$ и $c_{ij}=cos\theta_{ij}$, $\theta_{ij}$ -- угол смешивания, а $\delta_{CP}$ обозначает фазу нарушения CP-инвариантности.

\paragraph{\large Нейтринные осцилляции в среде}

При прохождении нейтрино через вещество оно может когерентно рассеиваться на частицах среды посредством реакций нейтрального и заряженного токов, что приводит к существенным отличиям от случая осцилляций в вакууме.

Влияние среды можно описать в терминах потенциалов, в которых распространяются нейтрино, зависящим от состава среды, электрической нейтральности, намагниченности (ориентации спинов), скоростей частиц среды.

При распространении через обычное вещество (среда электронейтральная,
немагнитная, нерелятивисткая) когерентное рассеяние $\nu_e$ на электронах и нуклонах отличается от рассеяния $\nu_{\mu}$ и $\nu_{\tau}$. В случае реакции $\nu_e \to \nu_e$ в такой среде используется эффективный потенциал $v(r)=\sqrt{2}\,G_F\,n_e(r)$, $G_F$ -- константа Ферми и $n_e(r)$ -- концентрация электронов на пути распространения нейтрино.

Как следствие, вероятности осцилляций модифицируются нетривиальным образом по механизму Михеева-Смирнова-Вольфенштейна (MSW). Эффекты материи особенно значимы на Солнце и других астрофизических объектах и событях, в частности, в коллапсах ядра сверхновых. 

Рассмотрим нейтрино $\nu_\alpha$, возникающее в среде в момент времени $t_0$. Состояние системы $\ket{\psi(t)}$ во время $t\geqslant t_0$ может быть выражено как $\ket{\psi(t)}=\sum_\beta \psi_\beta(t)\ket{\nu_\beta}$ с условием $\psi_\beta(t_0)=\delta_{\alpha\beta}$. Во внесистемных единицах ($\hbar=c=1$) и $t=r$ вероятность перехода нейтрино от состояния $\alpha$ в состояние $\beta$
\begin{equation}
\label{prob}
P_{\alpha\beta}(r)=|\psi_\beta(r)|^2.
\end{equation}

Как только нейтрино покидают среду, амплитуды эволюционируют в соответствии с уравнением, определяющим осцилляции в вакууме, решение которого проще, если записать их через массовые состояния. Согласно формуле (\ref{mixing}), обозначая через $\mathcal{A}_j=\phi_j(r_*)$ амплитуду вероятности обнаружения состояния аромата частицы $\nu_j$ на краю среды, для $r \geqslant r_*$ мы можем записать
\begin{equation}
\label{psi_b}
\psi_\beta(r)=\sum\limits_{j=1}^3 U_{\beta j}\, \mathcal{A}_j \exp(-iE_jL),
\end{equation}
где $E_j=\sqrt{|{\bf p}+m^2_j|}$ и $L=r-r_*$ -- расстояние, которое нейтрино преодолело в вакууме. Подставляя (\ref{psi_b}) в (\ref{prob}) получаем
\begin{equation}
P_{\alpha\beta}=\sum\limits_j |U_{\beta j}|^2 |\mathcal{A}_j|^2+2\sum\limits_{i>j}\text{Re}[U_{\beta i}U^*_{\beta j}\mathcal{A}_i\mathcal{A}^*_j\exp(-i\Delta_{ij}L)],
\end{equation}
с величинами $\Delta_{ij}=\Delta m^2_{ij}/(2E)$, где $E=|\bf p|$ и $\Delta m^2_{ij} \!= m^2_{i} - m^2_{j}$.

Проблема вычисления $P_{\alpha\beta}$ сводится к проблеме определения величины $\mathcal{A}_j$, то есть к определению амплитуд вероятности массовых состояний $\phi_j(r)$ в среде, с начальным условием $\phi_j(r_0)=U^*_{\alpha j}$. Для релятивистских нейтрино, распространяющихся в нормальном веществе, после вычитания глобальной фазы, эволюционное уравнение для этих амплитуд имеет вид
\begin{equation}
\label{Shr}
i\frac{d}{dr} \Phi(r)=[H_0+v(r)U^\dagger V U] \Phi(r),
\end{equation}
с $\Phi^T(r)=(\phi_1(r),\phi_2(r),\phi_3(r))$, $V=\text{diag}(1,0,0)$.  $H_0$ обозначает гамильтониан для эволюции в вакууме, в то время как второй член учитывает эффекты вещества из-за когерентного взаимодействия нейтрино с фоновыми частицами.

Уравнение (\ref{Shr}) можно переписать в более простой форме
\begin{equation}
i\frac{d}{dr}\Psi(r)=H(r)\Psi(r),\hspace{1cm} H(r)=H_0 + v(r)W,
\end{equation}
если сделать преобразования $\Psi(r)=\text{Г}^\dagger\Phi(r)$, $\text{Г}=\text{diag}(1,1,e^{i\delta_{CP}})$ и явный вид матрицы $W$:
\begin{equation}
\label{W}
W=
\begin{pmatrix}
c^2_{13}\, c^2_{12}& c_{12}\, s_{12}\, c^2_{13}& c_{12}\, c_{13}\, s_{13}\\
c_{12}\, s_{12}\, c^2_{13}& s^2_{12}\, c^2_{13}& s_{12}\, c_{13}\, s_{13}\\
c_{12}\, s_{13}\, c_{13}& s_{12}\, c_{13}\, s_{13}& s^2_{13}
\end{pmatrix}.
\end{equation}

\newpage 
\section{Постановка задачи}

%проблема, почему следует использовать разложение магнуса

\section{Разложение Магнуса}

Разложение Магнуса имеет важное значение, так как обеспечивает унитарность приближенных решений уравнения Шредингера. Оно представляет собой систематический способ построения аппроксимаций решения нестационарного уравнения Шредингера таким образом, чтобы в любом порядке оператор эволюции был унитарным.

\subsection{Теория}

Рассмотрим оператор эволюции $U(t,t_0)$, который переводит квантовомеханическое состояние (волновую функцию) $\Psi$ от времени $t_0$ ко времени $t$
%
\begin{equation}
\Psi(t)=U(t,t_0)\Psi(t_0),
\end{equation}
откуда следует очевидное условие $U(t_0,t_0)=I$, где $I$ -- единичный оператор.

С течением времени волновая функция $\Psi$ не должна менять свою норму~-- это константное значение, обеспечивающее сохранение вероятности. Математически это означает, что оператор $U(t,t_0)$ является унитарным 
%
\begin{equation}
U(t,t_0)\,U^\dagger(t,t_0)=U^\dagger(t,t_0)\,U(t,t_0)=I
\end{equation}
и удовлетворяет уравнению
%
\begin{equation}
\label{HU}
i\hbar\frac{\partial}{\partial t} U(t,t_0)=\lambda H(t)U(t,t_0),
\end{equation}
где $H(t)$ -- гамильтониан системы, которая в общем случае зависит от времени, а $\lambda$ -- промежуточный параметр, который в конце полагается $\lambda=1$.

Если бы мы не рассматривали пространство матриц, то решение (\ref{HU}) легко можно было бы найти с помощью иттераций
\begin{equation}
\label{U}
U(t,t_0)=I+\sum\limits_{n=1}^\infty\lambda^n P_n(t,t_0), \hspace{0.5cm} P_n(t,t_0)=0.
\end{equation}
Подставляя (\ref{U}) в уравнение (\ref{HU}) и приравнивая слагаемые при одинаковых степенях $\lambda$, возникает система уравнений на $P_n$
\begin{equation}
i\hbar\frac{\partial}{\partial t} P_1(t,t_0)=H(t), \hspace{0.5cm} i\hbar\frac{\partial}{\partial t} P_n(t,t_0)=H(t)\,P_{n-1}(t,t_0) \hspace{0.5cm} (n>1).
\end{equation}
интегрируем, получаем $P_n$
\begin{equation}
\label{P_n}
P_n(t,t_0)=\biggl(-\frac{i}{\hbar}\biggr)^n\int\limits_{t_0}^t dt_1\int\limits_{t_0}^{t_1}dt_2\cdots\int\limits_{t_0}^{t_{n-1}}\!\!dt_n \, H_1 H_2\ldots H_n, \hspace{1cm} H_i\equiv H(t_i). 
\end{equation}
Если гамильтониан системы не зависит от времени, то $P_n$ приобретает компактный вид и ряд (\ref{U}) сводится к экспоненте
\begin{equation}
P_n = \frac{1}{n!}\biggl(-i(t-t_0)H/\hbar\biggr)^n \;\; \to \;\; U(t,t_0)=\exp\{-i(t-t_0)\lambda H/\hbar\}.
\end{equation}

Так как мы ищем решение матричного уравнения, то (\ref{P_n}) будет верным лишь в том случае, если $[H(t_1),H(t_2)]=0, \; \forall t_1,\, t_2$. В общем случае это не так для $H=H(t)$.

Для удобства сделаем переопределение: $A(t)\equiv -iH(t)/\hbar$, оператор $A(t)$ является антиэрмитовым. Переписываем матричное уравнение
\begin{equation}
\frac{\partial}{\partial t} U(t,t_0)=\lambda A(t) U(t,t_0).
\end{equation} 

Учтем, что работаем с матрицами и будем искать решение в виде ряда:
\begin{equation}
U(t,t_0)=\exp\{\Omega(t,t_0)\}=I+\sum\limits_{n=1}^\infty \frac{1}{n!}\,\Omega^n, \hspace{0.5cm} \Omega(t_0,t_0)=0,
\end{equation}
если $A\ne A(t)$, то $\Omega(t,t_0)=\lambda(t-t_0)A(t)$.

Так как обычным дифференциальным правилам (для функций) матричные экспоненты не подчиняются, то встает вопрос, как получить дифференциальное уравнение на $\Omega(t,t_0)$. Чтобы его составить, нужно воспользоваться 
\begin{itemize}
	\item[1)] групповым свойством оператора эволюции
	$$
	U(t_2,t_0)=U(t_2,t_1)\,U(t_1,t_0);
	$$
	\item[2)] формулой Бейкера — Кемпбелла — Хаусдорфа для матричных экспонент
	$$
	\exp(X)\exp(Y)=\exp\biggl(X+Y+\frac{1}{2}[X,Y]+\frac{1}{12}[X,[X,Y]]+\frac{1}{12}[Y,[Y,X]]+\cdots\biggr),
	$$
	а точнее, его приближением
	$$
	\exp(X)\exp(Y) \simeq \exp\biggl(X+Y+\sum\limits_{k=1}^\infty (-1)^n \frac{B_n}{n!}\underbrace{[Y,[\cdots[Y}_{\mbox{n \;\text{раз}}},X]]\cdots]+\mathcal{O}(X^2)\biggr),
	$$
	где  $B_n$ -- числа Бернулли, а квадратные скобки обозначают коммутатор $[X,Y]=XY-YX$ для матриц одинаковых размерностей..
\end{itemize}

Используя 1), где $t_2=t+\delta t$, $t_1=t$ и предполагая, что $A(t)\simeq A(t+\delta t)$, получаем уравнение
\begin{equation}
e^{\Omega(t+\delta t,t_0)}\simeq e^{\lambda A(t)\delta t}\!\cdot e^{\Omega(t,t_0)},
\end{equation}
и, применяя к нему 2) в пределе $\delta t \to 0$, можно получить дифференциальное уравнение на $\Omega(t,t_0)$
\begin{equation}
\label{diffO}
\frac{\partial}{\partial t} \Omega(t,t_0) = \lambda A(t) + \lambda \sum\limits_{n=1}^\infty (-1)^n \frac{B_n}{n!}\underbrace{[\Omega(t,t_0),[\cdots[\Omega(t,t_0)}_{\mbox{n раз}},A(t)]]\cdots].
\end{equation}

Выигрыш с экспоненциальным представлением оператора эволюции появляется, когда $\Omega$ выражается в виде ряда по степеням $\lambda$, который называется рядом Магнуса $\Omega=\sum\limits_{n=1}^\infty \lambda^k \Omega_k$. Подставляем его  в (\ref{diffO}), приравниваем слагаемые при одинаковых степенях $\lambda$ и, после некоторых вычислений, первые три члена ряда Магнуса:

\begin{align}
\label{On1}
\Omega_1(t,t_0)&=\int\limits_{t_0}^t dt_1\, A_1, \\
\label{On2}
\Omega_2(t,t_0)&=\frac{1}{2}\int\limits_{t_0}^t dt_1 \int\limits_{t_0}^{t_1} dt_2\, [A_1,A_2],\\
\label{On3}
\Omega_3(t,t_0)&=\frac{1}{6}\int\limits_{t_0}^t dt_1 \int\limits_{t_0}^{t_1} dt_2 \int\limits_{t_0}^{t_2} dt_3\, \biggl\{[A_1,[A_2,A_3]]+[[A_1,A_2],A_3]\biggr\},
\end{align}

где, как и прежде, использовано обозначение $A_i\equiv A(t_i)$.

Важным моментом является то, что при любом порядке по $\lambda $ усеченная сумма ряда Магнуса всегда антиэрмитова, так как $A(t)$ -- антиэрмитовый оператор и коммутатор антиэрмитовых операторов так же антиэрмитов. Следовательно, экспонента такого ряда будет всегда давать унитарное приближение для $U(t,t_0)$.

Таким образом решение нестационарного уравнения Шрёдингера 
\begin{equation}
i\hbar\frac{d}{d t}\Psi(t)=H(t)\Psi(t), 
\end{equation}
или, в терминах оператора $A(t)$
\begin{equation}
\label{dPsi}
\frac{d}{d t}\Psi(t)=A(t)\Psi(t), \hspace{1cm} A \in \mathbb{C}^{n\times n}, \hspace{0.5cm} \Psi \in \mathbb{C}^n,
\end{equation}
даётся следующей формулой
\begin{equation}
\Psi(t)=\exp(\Omega(t,t_0))\Psi(t_0), \hspace{1cm} \Omega(t,t_0) \in \mathbb{C}^{n\times n}, \hspace{0.5cm} \Omega(t_0,t_0)=0,
\end{equation}
где $\Omega(t,t_0)=\sum\limits_{n=1}^\infty\Omega_n(t,t_0)$ с начальным условием $\Omega_n(t_0,t_0)=0$, и три первых члена этого ряда (\ref{On1}-\ref{On3}).

\subsection{Численная реализация}

Существуют некоторые естественные ограничения на численный счет, например, посчитать точно бесконечные ряды не представляется возможным, из-за чего возникает проблема счёта матричной экспоненты. Необходимо прибегать к наиболее оптимальным вариантам приближенных вычислений.

В целях использования разложения Магнуса в качестве числового интегратора, который дает решение $\Psi(t)$, начиная с $\Psi(t_0)$, вопрос сосредоточен на том, как эффективно обрабатывать один шаг интегрирования.

Отдельные точки, в которых вычисляется решение на интервале $[t_0,\,t_f], \;\;t_0 < t_1 < \cdots < t_N = t_f$  связаны с приращением времени на величину $h_n=t_{n+1}-t_n$, где $0\leqslant n \leqslant N-1$. Определим решение в точке $t_{n+1}$
\begin{equation}
\Psi(t_{n+1})=\exp\{\Omega(t_n+h_n,t_n)\}\Psi(t_n).
\end{equation}

После всех иттераций, решение в конечной точке интервала даёт
\begin{equation}
\Psi(t_f)=\prod\limits_{n=0}^{N-1} \exp\{\Omega(t_n;h_n)\}\Psi(t_0),
\end{equation}
с сокращенным обозначением $\Omega(t_n;h_n)\equiv \Omega(t_n + h_n,t_n)$.

\subsubsection{Нейтринные осцилляции в среде}

Дифференциальная система, управляющая эволюцией состояний в трёх-нейтринном случае осцилляций в среде является одним из типов уравнений (\ref{dPsi}), где $n=3$, $t=r$, и $A\equiv -iH$
\begin{equation}
i\frac{d}{dr}\Psi(r)=H(r)\Psi(r),
\end{equation}
где $H(r)=H_0 + v(r)W$, а явный вид матрицы $W$ в (\ref{W}).

Мы будем решать на примере, где начальное состояние соответствовало электронному нейтрино
\begin{equation}
\Psi(r_0)=
\begin{pmatrix}
c_{12}\, c_{13}\\
s_{12}\, c_{13}\\
s_{13} 
\end{pmatrix}.
\end{equation}

Мы предполагаем нормальную иерархию масс и принимаем наиболее подходящими значения параметров осцилляции в трёхнейтринном случае, полученные из текущих данных по нейтрино %ссылка
: $\Delta m^2_{21}=7.54\times 10^{-5}\, \text{эВ}^2,\, \Delta m^2_{31}=2.47 \times 10^{-3}\, \text{эВ}^2,\, \sin^2\theta_{12}=0.308,\, \sin^2\theta_{23}=0.437$ и $\sin^2\theta_{13}=0.0234$.

Переходим к безразмерным переменным $\xi\equiv r/R_\odot$, выражая расстояние в единицах солнечного радиуса $R_\odot=6.96\times10^5$ км, и используем матрицу $H_0$ в виде
\begin{equation}
H_0=\frac{a}{\mathcal{E}}
  \begin{pmatrix}
  0& 0& 0\\
  0& b& 0\\
  0& 0& 1 
  \end{pmatrix},
\end{equation}
здесь $\mathcal{E}$ это численное значение энергии нейтрино в МэВ и $a=4.35196\times 10^6$ и $b=0.030554$ безразмерные параметры.

Обозначим за $r$ порядок точности решения по степеням $h_n$. Другими словами, мы будем находить решение в точке $\Psi(\xi_n+h_n)$ с точностью до $\mathcal{O}(h^{r+1}_n)$. За этот порядок отвечает то, каким численным методом мы будем находить $\Omega_k$ и каким конечным значением $k$ в ряде Магнуса мы ограничимся. Приближение к усеченному ряду Магнуса порядка $r$: $\Omega(\xi_n;h_n)\simeq \Omega^{[r]}(\xi_n;h_n)$.

Таким образом, решение порядка точности $r$ по $h_n$ будет реализовано следующей формулой
\begin{equation}
\Psi(\xi_{n+1})=\exp\bigl\{\Omega^{[r]}(\xi_n;h_n)\bigr\}\Psi(\xi_n).
\end{equation}
 
\subsubsection{Методы}

Рассмотрим два метода реализации, где $r=2,4$, названных соответственно M2 и M4.

\paragraph{M2}

Метод приближения второго порядка является самым простым, так как из ряда Магнуса остаётся только первое слагаемое $\Omega^{[2]}=\Omega_1$
\begin{equation}
\label{Omega_1}
\Omega_1(\xi_n;h_n)=-i\!\int\limits_{\xi_n}^{\xi_n+h_n}\! dt H(t),
\end{equation} 
и для достижения заданной точности достаточно одной точки, чтобы оценить значение интеграла (\ref{Omega_1}).

Таким образом, второй порядок точности реализуется формулой
\begin{equation}
\Omega^{[2]}(\xi_n;h_n)=-iH(\bar\xi)h_n=-i(H_0+\bar{v}\,W)h_n,
\end{equation}
где взята средняя точка интервала интегрировани $\bar\xi\equiv\xi_n+h_n/2$. Величина $\bar{v}\equiv v(\bar\xi)$ пересчитывается на каждом шаге.

\paragraph{M4}

Для достижения $r=4$ необходимо считать первые два члена разложения $\Omega^{[4]}=\Omega_1 + \Omega_2$, а аппроксимировать интегралы двухточечными квадратурами Гаусса-Лежандра.

Точки вычисления квадратур
\begin{equation}
\xi_{\pm}=\xi_n+\bigl(1\pm\frac{1}{\sqrt{3}}\bigr)\frac{h_n}{2},
\end{equation} 
в которых введём величины $H_{\pm}=H(\xi_{\pm})$.

Конечная формула такого метода сводится к следующему выражению 
\begin{equation}
\label{Omega4}
\Omega^{[4]}(\xi_n;h_n)=-i(H_++H_-)\frac{h_n}{2}+\frac{\sqrt{3}}{12}[H_-,H_+]h^2_n.
\end{equation}
В качестве альтернативы квадратуры Симпсона также дают эквивалентное приближение четвертого порядка.

\section{Модели}

Эффекты материи играют важную роль на Солнце и в коллапсах сверхновых, первым делом необходимо рассмотреть именно эти модели и обратить внимание на проблемы их численной реализации при использовании разложения Магнуса. Непосредственно модель задаётся функцией $v(\xi)$.

\subsection{Солнце}

Очень важно рассмотреть модель Солнца, так как оно является источником большого количества нейтрино, приходящих на Землю. Полный поток солнечных нейтрино оценивается величиной $N_\nu\simeq 1.8 \times 10^{38} \frac{\text{нейтрино}}{\text{сек}}$.

В Солнце электронная плотность хорошо аппроксимируется экспоненциальным профилем
\begin{equation}
v(\xi)=\gamma \exp(-\eta\xi),
\end{equation}
с $\gamma=6.5956\times10^4$ и $\eta=10.54$. Интервал интегрирования в единицах солнечного радиуса был $\xi\in[0.1,\,1]$.

Если в (\ref{Omega4}) раскрыть коммутатор, подставляя в явном виде $H(\xi_\pm)=H_0+v(\xi_\pm)W$, то там возникает разность $(v_+-v_-)$, $v_\pm\equiv v(\xi_\pm)$. В этом месте может происходить потеря точности, так как при вычитании близких чисел значимые разряды могут потеряться, что может в разы увеличить относительную погрешность.

Экспоненциальная функция с отрицательным показателем быстро падает, а при достаточно малом шаге (расчеты показывают, что оптимальный вариант для $h_n\sim 10^{-5}-10^{-6}$) это может сильно сказаться на результате.

Мы используем числа формата double, мантисса способна хранить 15 старших десятичных значащий цифр. Когда вычитаются близкие по модулю числа, старшие биты мантиссы обнуляются, что компенсируется её сдвигом и изменением экспоненты. При этом ненулевые младшие биты мантиссы становятся старшими, а новые младшие биты неизвестны (так как для их хранения в исходных числах не хватило места) и будут заполнены нулями. И если потом результат подобного вычитания умножить на очень большое число, то те неизвестные младшие биты сдвинутся в сторону старших, приобретая случайные \glqq шумовые\grqq $\,$ значения.

\subsection{Сверхновая}

Взрыв сверхновой является также важным источником нейтрино. Во время взрыва 99\% энергии гравитационной связи звезды (порядка $3\cdot10^53$ эрг) высвобождается в виде нейтрино и антинейтрино всех ароматов.

В случае сверхновой была использована
\begin{equation}
v(\xi)=\gamma/\xi^3,
\end{equation}
с $\gamma=52.934$. Интегрирование велось в интервале $[0.02,\,20]$.

Она падает не так быстро, как в случае с солнечной моделью, поэтому разность $(v_+-v_-)$ не приведёт к нарушению заданной точности счета. В этом смысле, модель сверхновой считается несколько точнее, чем модель Солнца.


\section{Изменение шага}

Самый простой способ реализовать рассмотренные методы интегрирования -- использовать постоянное значение шага, то есть разбить интервал интегрирования на равные промежутки $h_n=h=(\xi_f-\xi_0)/N$ и после установить приращение $\xi_n=\xi_0+nh$. Эта реализация не является эффективной, так как решение $\Psi(\xi)$ может испытывать быстрые изменения вдоль эволюции на некоторых промежутках и медленно развиваться в других, что может заметно отразиться на конечном решении. Наиболее оптимально использовать шаг, который регулируется автоматически в процессе счета наиболее подходящим образом.
Один из возможных способов реализации -- ввести условие, чтобы локальная ошибка была ниже установленного значения tol (чувствительности счета), если это не так -- следует уменьшить шаг.

Для оценки локальной ошибки $E_r$ в точке $\xi_{n+1}$  понадобятся значения обоих рассмотренных методов
\begin{equation}
\hat\Psi_{n+1}=e^{\Omega^{[2]}(\xi_n,h_n)}\Psi_n,
\hspace{1cm}
\Psi_{n+1}=e^{\Omega^{[4]}(\xi_n,h_n)}\Psi_n,
\end{equation}
М2 и М4 соответственно. Тогда локальную ошибку метода М2 можно выразить следующим образом
\begin{equation}
\label{Er}
E_r=\parallel \hat\Psi_{n+1} - \Psi_{n+1} \parallel,
\end{equation}
где мы использовали эвклидовскую норму вектора $\parallel\! X\!\parallel=\sqrt{\sum\limits_i x^2_i}$.

Таким образом, если в данной точке $E_r>\text{tol}$, то интегратор возвращается на шаг назад и считает заново в этой точке с новым меньшим шагом $h_{new}$
\begin{equation}
h_{new}=s\,h_c\, \biggl(\frac{\text{tol}}{E_r}\biggr)^{1/3},
\end{equation}
где $h_c$ означает текущее значение шага и $s$ (safety factor) обеспечивает уменьшение вероятности того, что при следующем шаге опять сработает условие $E_r>\text{tol}$. В данной работе использовано значение $s=0.8$.

Так как самая затратная по времени часть работы программы заключена в вычислении матричных экспонент, то прямая оценка $E_r$ как (\ref{Er}) может значительно увеличить общее время вычислительной работы алгоритма. Чтобы избежать этого, мы можем выразить
\begin{equation}
\hat\Psi_{n+1}-\Psi_{n+1}=(e^{\Omega^{[2]}-e^{\Omega^{[4]}}})\Psi_n=(e^Z-I)\Psi_{n+1},
\end{equation}
где 
\begin{equation}
\label{Z}
Z=\ln(e^{\Omega^{[2]}}\,e^{-\Omega^{[4]}})=\Omega^{[2]} - \Omega^{[4]} - \frac{1}{2}[\Omega^{[2]},\Omega^{[4]}]+\cdots .
\end{equation}
Используя это, можно посчитать локальную ошибку приближенно и менее времязатратно следующим образом
\begin{equation}
E_r\simeq\parallel(h^2_n S_1+h^3_n S_2 + \frac{1}{2}h^4_n S^2_1)\Psi_{n+1}\parallel + \mathcal{O}(h^5_n),
\end{equation}
где введены обозначения
\begin{align}
S_1&=-\frac{\sqrt{3}}{12}(v_+-v_-)[H_0,W],\\
S_2&=i\frac{\sqrt{3}}{24}(v_+-v_-)\Bigl([H_0,[H_0,W]]+\frac{1}{2}(v_++v_-)[W,[H_0,W]]\Bigr)
\end{align}


\section{Код} %? tol, 64 бит

Для того, чтобы в некоторых местах не терять точность и сделать код наиболее эффективным, необходимо было учитывать особенности численной реализации формул. 

\lstset{language=C,%
	basicstyle=\ttfamily,%
	keywordstyle=\color{red},%
	commentstyle=\small\color{blue}}
\begin{lstlisting}[texcl]


\end{lstlisting}


\section{Обсуждение}

\end{document}